\section{Experiments}
\label{sec:experiments}
% (recommended size: 1.5 pages)
	In this section, a realistic simulation of clients connecting/disconnecting from the server is first described. This is one of the requirements of the customer WantGame BV. After that, the experimental setup used is provided. Finally, the experiments that have been conducted on the Fault-tolerant Distributed Dragon Game (FDDG) and the outcome of these experiments are presented.
	
\subsection{Simulation of connecting/disconnecting clients}
\label{subsec:simulation_clients}
One of the requirements of our game, is that a realistic game trace should be used to simulate the behavior of clients connecting and disconnecting to and from the server.
A potential data set with game traces is described in the paper of Guo et al. \cite{guo2012game}.
Here, the design of the GTA (Game Trace Archive) is explained and possible applications of the data are discussed.
Amongst the genres of the analyzed games are Massively Multiplayer Online Role-Playing Game (MMORPGs), board games and Real-time Strategy (RTS) games. 
Since FDDG closely resembles a MMORPG, the WoWAH game trace \cite{lee2011world} was used.
This game trace archive contains data about 91065 players during 1107 days. The dataset is available as a free download on the author's website\footnote{http://mmnet.iis.sinica.edu.tw/dl/wowah/}.
From this dataset, the number of online players during 24 hours was extracted to use in the simulations. 
A visual representation of the amount of online avatars in the game trace can be found in Figure \ref{fig:online_players_plot}. Here, one can see that the number of online players is minimal during the morning and maximal around midnight.

\begin{figure}[h!]
  \centering
    \includegraphics[width=\textwidth]{images/online_players_plot}
    
  \caption{The number of online players during 24 hours (2007-07-14)}
  \label{fig:online_players_plot}
\end{figure}

Also, the number of players connecting and disconnecting is tracked during 10-minute time intervals.
This interval matches the interval being used in the WoWAH dataset.
Using this data, a realistic pattern of players connecting and disconnecting to and from the game could be constructed.
A simulation file has been created, containing information about all events during 24 hours. This file can be used to run the simulation of connecting and disconnecting players.

\subsection{Experimental setup}
\label{subsec:experimental_setup}
% Describe the working environments (DAS-4, Amazon EC2,
% etc.), the general workload and monitoring tools and libraries, other tools and
% libraries you have used to implement and deploy your system, other tools and
% libraries used to conduct your experiments.

To experiment with FDDG and inspect the performance, the game was simulated on the DAS-4 supercomputer.
The common setup of the experiments consisted of 5 servers where each server runs on one of the DAS-4 nodes. Furthermore, we have two nodes with 50 client running on each node.
By running the simulations on the DAS-4, it is guaranteed that the game was tested in a truly distributed system, where processes were running on physical different nodes.
The disadvantage of running the game on the DAS-4 is that we cannot see the GUI that allows for easier debugging and observation of the game state.
Furthermore, a self-created message system that makes use of \emph{actions} was implemented to send updates and acknowledgments inside the game.
A simulation starts as soon as the first client joins the server. The game is then considered active and will run until either all players are dead or all dragons are slain.


\subsection{Experiments}
\label{subsec:experiments}
% Describe the experiments you have conducted to analyze each
% system feature, such as consistency, scalability, fault-tolerance, and
% performance. Analyze the results obtained for each system feature. Use one
% sub-section per experiment (or feature). In the analysis, also report:
% i. Service metrics of the experiment, such as runtime and response time of
% the service, etc.
% ii. (optional) Usage metrics and costs of the experiment.
	Experiments were ran concerning the consistency, fault-tolerance and scalability of the game. Below are the results of the experiments conducted for each of these features. Moreover, an analysis of the number of messages in the game was made as well.

	\subsubsection{Consistency}
	\label{subsubsec:consistency}
		In order to test consistency, a simulation was run with exactly the same setup multiple times.
		It was checked that the result was consistent, that is the same field state at the end of each simulation. 
		
	\subsubsection{Fault-tolerant}
	\label{subsubsec:fault-tolerant}
		To test fault-tolerance, the following scenarios are considered:
		\begin{enumerate}
			\item During simulations, some some client nodes are killed. The servers should then remove the players from the field and continue the game.
			\item During the simulation, one server is killed. Clients should connect to another server in this case.
		\end{enumerate}
		
		More information about the design of fault tolerance in the system can be found in section \ref{subsubsec:client_server_crashes}. If a client crashes, the server should notice that after two failed heartbeats to this client. During the simulations, the event that a server sends two failed heartbeats to the client is logged. Using the GUI, it is possible to determine whether the player is really removed from the server. This turned out to be the case if clients are crashed on purpose. The servers remove the disconnected player from the field and continue the game.\\
		When a server fails, clients should immediately reconnect to one of the other available servers. When killing a server, it was noticed that the clients connected to this server, were trying to select another server that is still available. Only if all servers are down, clients will stop trying to connect to one of the servers.
		
	\subsubsection{Scalability}
	\label{subsubsec:scalability} 
		One of the requirements by WantGame is that the system should be scalable. It is their aim to run 5 servers and allow 100 clients to connect with these servers.
		To test this setting, 5 servers were ran on 5 nodes of the DAS-4 supercomputer and 100 clients in bunches of 50 on 2 nodes. Here, it is concluded that the design is suitable for this setup and meets the requirement set by WantGame.
		Please note that no tests were performed that went beyond this setting, but it is believed that the limit was not yet reached. Testing the scalability further is recommended for future work.
		
	\subsubsection{Performance}
	\label{subsubsec:performance}
		To test the performance of the system, the response times between a client requesting an action and the server acknowledging this request (after having communicated with its peers) was measured. 
		The results of this experiment are visible in Figure \ref{fig:boxplot_response_times}.
		The first observation is that almost all of the requests took no more than $\pm$30 ms which is quite satisfactory. Since all nodes ran in the same cluster, this number may be higher when nodes are physically distributed.
		The next observation are the outliers which can reach up to $\pm$500 ms. 
		This may be due to servers getting a lot of request simultaneously, which cause some delay.
		Since FDDG is a MMORPG, 500 ms is an acceptable worst case of response time.
		
		\begin{figure}[h!]
		  \centering
		    \includegraphics[width=0.6\textwidth, height=0.59\textheight]{images/boxplot_response_times}
		    
		  \caption{A boxplot visualizing the response times of actions between clients and server.}
		  \label{fig:boxplot_response_times}
		\end{figure}
		
	\subsubsection{Message complexity of the system}
	\label{subsubsec:nummessages}
		The chosen consistency model in the game has great impact on the number of messages between servers and clients. 
		In this subsection, an analysis of the total number of messages in the system is performed and simulations of clients and servers are performed to determine the total number of required messages.\\
		
		% THIS IS ALREADY EXPLAINED IN THE DESIGN.... WHAT DO WE DO WITH IT?
		
		First, the number of messages required for a client to perform an action was analyzed. 
		A client performing an action, sends this action to the server it is connected to. 
		The server sends this request to all other servers and these servers respond with an acknowledgment if the action is valid. 
		Finally, the server let all other servers know this action can be performed. 
		The action is then performed and send to all clients. 
		More information about this process can be found in section \ref{subsubsec:clients_actions}. 
		The total number of messages required for one action can be described by the following formula ($ n $ is the total number of messages, $ s $ is the number of servers, $ c $ is the number of clients):
		$$ n = 3(s - 1) + c + 1 $$
		The formula above excluded heartbeat messages sent between clients and servers.
		This formula is used to make an approximation of the amount of messages that has to be sent between the servers and clients. 
		Suppose $ s = 5 $ and $ c = 100 $. 
		If every client performs an action every second, a total of 678.000 messages or 6780 messages per client have to be exchanged. 
		This number can be reduced by switching to another model for consistency.\\
		To gain more insight in the number of messages, a setup with 5 servers and 20 dragons was run. 
		In two runs, one with 50 clients and one with 100 clients, it took around one minute to finish the game (in both games all players were killed). 
		With 50 clients, the total number of messages was around 5600 from the beginning to the end of the game. 
		With 100 clients, a total of 14900 messages between servers and clients are exchanged. 
		The fact that this number is way below the approximated number of messages in the system can be explained by the fact that not all clients were active during the total duration of the game. 
		Moreover, players that have died during the game, ca no longer performing actions. We also noticed that the number of messages sent by a server to clients are about 2.5 times larger than the number of messages to other servers.\\
		The final analysis discussed here is the number of heartbeats between clients and servers. 
		With five servers, each server received and sent around 150 heartbeats. 
		The total number of heartbeats sent to clients is for each server around 105. 
		We can conclude that these numbers do no contribute significantly to the amount messages in the system.
