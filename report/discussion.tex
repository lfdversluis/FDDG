\section{Main findings \& trade-offs}
% (recommended size: 1 page): 

% Summarize the main findings of your work
% and discuss the tradeoffs inherent in the design of the DAS system. Should WantGame
% use a distributed system to implement the DAS system? Try to extrapolate from the
% results reported in Section 6.b for system workloads that are orders of magnitude
% higher than what you have tried in real-world experiments.

	As discussed earlier in section \ref{subsec:system_overview}, we decided to use a mirrored game design with multiple servers. 
	Since the field has is 25x25 (625 points), storing the entire field doesn't require much storage. 
	Moreover, since every server maintains the whole board, a server crash can be easily overcome.
	Once the crashed server recovers, one of the operational servers can send the entire field to that server. 
	The crashed server can then initialize itself and immediately participate in the system again. 
	As result this also means that the current design is highly fault tolerant as we can tolerate crashes of all but one server simultaneously. 
	Only if all servers crash simultaneously, data and game state will be unrecoverable.
	We can conclude that the current setting offers WantGame a consistent and available system. 
	The drawbacks of this approach is the loss of performance due to an increased amount of messages being sent. This is because of actions being broadcast to all servers and clients in order to maintain a mirrored game state.
	
	TODO results from 6b. (experiments).